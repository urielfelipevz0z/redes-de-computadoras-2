% Configuración personalizada de cajas para documentos técnicos
% Usando tcolorbox con iconos FontAwesome

\usepackage{tcolorbox}
\usepackage{fontawesome5}

% Definir colores personalizados
\definecolor{infocolor}{RGB}{58, 135, 173}
\definecolor{warningcolor}{RGB}{245, 166, 35}
\definecolor{errorcolor}{RGB}{208, 2, 27}
\definecolor{successcolor}{RGB}{40, 167, 69}

% Caja de información
\newtcolorbox{info-box}{
  colback=infocolor!10,
  colframe=infocolor,
  arc=2pt,
  outer arc=2pt,
  leftrule=4pt,
  rightrule=0pt,
  toprule=0pt,
  bottomrule=0pt,
  left=8pt,
  right=8pt,
  top=8pt,
  bottom=8pt,
  before upper={\parindent15pt\noindent\textcolor{infocolor}{\Large\faInfoCircle}\enspace}
}

% Caja de advertencia
\newtcolorbox{warning-box}{
  colback=warningcolor!10,
  colframe=warningcolor,
  arc=2pt,
  outer arc=2pt,
  leftrule=4pt,
  rightrule=0pt,
  toprule=0pt,
  bottomrule=0pt,
  left=8pt,
  right=8pt,
  top=8pt,
  bottom=8pt,
  before upper={\parindent15pt\noindent\textcolor{warningcolor}{\Large\faExclamationTriangle}\enspace}
}

% Caja de error
\newtcolorbox{error-box}{
  colback=errorcolor!10,
  colframe=errorcolor,
  arc=2pt,
  outer arc=2pt,
  leftrule=4pt,
  rightrule=0pt,
  toprule=0pt,
  bottomrule=0pt,
  left=8pt,
  right=8pt,
  top=8pt,
  bottom=8pt,
  before upper={\parindent15pt\noindent\textcolor{errorcolor}{\Large\faTimes}\enspace}
}

% Caja de éxito
\newtcolorbox{success-box}{
  colback=successcolor!10,
  colframe=successcolor,
  arc=2pt,
  outer arc=2pt,
  leftrule=4pt,
  rightrule=0pt,
  toprule=0pt,
  bottomrule=0pt,
  left=8pt,
  right=8pt,
  top=8pt,
  bottom=8pt,
  before upper={\parindent15pt\noindent\textcolor{successcolor}{\Large\faCheck}\enspace}
}

% Configuración de highlighting para código Cisco IOS
\lstdefinelanguage{cisco-ios}{
  keywords={interface, ip, router, switch, vlan, access, trunk, enable, configure, terminal, show, clear, debug, no, shutdown, description, switchport, mode, spanning-tree, portfast, line, con, vty, login, password, secret, hostname, banner, motd, end},
  keywordstyle=\color{blue}\bfseries,
  ndkeywords={FastEthernet, GigabitEthernet, Serial, Loopback, VLAN, address, route, protocol, security, range, default-gateway},
  ndkeywordstyle=\color{darkgray}\bfseries,
  identifierstyle=\color{black},
  sensitive=false,
  comment=[l]{\#},
  comment=[l]{!},
  commentstyle=\color{gray}\ttfamily,
  stringstyle=\color{red}\ttfamily,
  morestring=[b]',
  morestring=[b]"
}

% Estilo para código bash/shell
\lstdefinelanguage{bash}{
  keywords={sudo, apt, update, install, which, dpkg, ping, echo, nc, tshark, wireshark, macof, grep, cat, head, tail, chmod, cp, mv, rm, ls, cd, mkdir},
  keywordstyle=\color{blue}\bfseries,
  ndkeywords={-c, -i, -f, -u, -l, -y, --version},
  ndkeywordstyle=\color{purple}\bfseries,
  identifierstyle=\color{black},
  sensitive=true,
  comment=[l]{\#},
  commentstyle=\color{gray}\ttfamily,
  stringstyle=\color{red}\ttfamily,
  morestring=[b]',
  morestring=[b]"
}

% Configuración general de listings - corregir font size
\lstset{
  basicstyle=\footnotesize\ttfamily,
  backgroundcolor=\color{gray!10},
  frame=single,
  framerule=0pt,
  frameround=tttt,
  rulecolor=\color{gray!30},
  numbers=left,
  numberstyle=\tiny\color{gray},
  stepnumber=1,
  numbersep=8pt,
  showstringspaces=false,
  breaklines=true,
  breakatwhitespace=false,
  tabsize=2,
  captionpos=b,
  % Configuración para evitar overflow de páginas
  breaklines=true,
  prebreak=\raisebox{0ex}[0ex][0ex]{\ensuremath{\hookleftarrow}},
  frame=single,
  framesep=3pt,
  xleftmargin=3pt,
  xrightmargin=3pt
}

% Configuración para mantener proporción de imágenes
\usepackage{graphicx}
\makeatletter
\def\maxwidth{\ifdim\Gin@nat@width>\linewidth\linewidth\else\Gin@nat@width\fi}
\def\maxheight{\ifdim\Gin@nat@height>0.8\textheight 0.8\textheight\else\Gin@nat@height\fi}
\makeatother
% Configuración global para todas las imágenes
\setkeys{Gin}{width=\maxwidth,height=\maxheight,keepaspectratio}
