\usepackage{tcolorbox}
\tcbuselibrary{skins,listings,breakable}
\usepackage{fontawesome5}
\usepackage{etoolbox}

\usepackage{graphicx}
\let\oldincludegraphics\includegraphics
\renewcommand{\includegraphics}[2][]{%
  \oldincludegraphics[width=\linewidth,#1]{#2}%
}

% Colores para cajas de notificación
\definecolor{infocolor}{RGB}{58, 135, 173}
\definecolor{warningcolor}{RGB}{245, 166, 35}
\definecolor{errorcolor}{RGB}{208, 2, 27}
\definecolor{successcolor}{RGB}{40, 167, 69}

% Colores para terminal Cisco IOS (alto contraste)
\definecolor{ciscoframe}{RGB}{0, 51, 102}
\definecolor{ciscobg}{RGB}{248, 252, 255}
\definecolor{ciscotext}{RGB}{0, 0, 0}
\definecolor{ciscoblue}{RGB}{0, 0, 255}
\definecolor{ciscogreen}{RGB}{0, 128, 0}
\definecolor{ciscopurple}{RGB}{128, 0, 128}
\definecolor{ciscored}{RGB}{220, 20, 60}

% Colores para terminal Linux/Bash (tema Ubuntu/Gnome Terminal clásico)
\definecolor{linuxframe}{RGB}{48, 10, 36}
\definecolor{linuxbg}{RGB}{48, 10, 36}
\definecolor{linuxwhite}{RGB}{255, 255, 255}
\definecolor{linuxcyan}{RGB}{6, 152, 154}
\definecolor{linuxgreen}{RGB}{78, 154, 6}
\definecolor{linuxyellow}{RGB}{196, 160, 0}
\definecolor{linuxblue}{RGB}{52, 101, 164}
\definecolor{linuxgray}{RGB}{211, 215, 207}
\definecolor{linuxpromptgreen}{RGB}{138, 226, 52}
\definecolor{linuxpromptblue}{RGB}{114, 159, 207}
\definecolor{linuxtitlebar}{RGB}{87, 64, 80}

\lstdefinestyle{custom-cisco-style}{
  language=,
  basicstyle=\ttfamily\footnotesize\color{ciscotext},
  backgroundcolor=\color{ciscobg},
  frame=none,
  numbers=none,
  showstringspaces=false,
  breaklines=true,
  breakatwhitespace=true,
  tabsize=2,
  captionpos=none,
  aboveskip=0pt,
  belowskip=0pt,
  xleftmargin=0pt,
  xrightmargin=0pt,
  framexleftmargin=0pt,
  framexrightmargin=0pt,
  % Palabras clave principales de Cisco IOS
  morekeywords={
    enable, configure, terminal, interface, ip, router, switch, vlan, 
    show, clear, debug, no, shutdown, exit, end, copy, reload, write,
    access, trunk, switchport, mode, spanning-tree, portfast,
    line, con, vty, login, password, secret, hostname, banner, motd,
    running-config, startup-config, memory, flash, tftp,
    port-security, maximum, violation, aging, sticky, mac-address,
    count, dynamic, static, learned, secure, restrict, protect
  },
  keywordstyle=\bfseries\color{ciscoblue},
  % Palabras clave secundarias (interfaces, protocolos)
  morekeywords=[2]{
    FastEthernet, GigabitEthernet, Serial, Loopback, VLAN,
    route, protocol, range, default-gateway, version, brief, detail,
    trunk, native, allowed, auto, full, half, dot1q, active, passive,
    UP, DOWN, administratively, connected, static, dynamic
  },
  keywordstyle=[2]\bfseries\color{ciscopurple},
  % Strings y valores
  stringstyle=\color{ciscored},
  morestring=[b]",
  morestring=[b]',
  % Comentarios (líneas que empiezan con !)
  morecomment=[l]{!},
  commentstyle=\color{ciscogreen}\itshape,
  % Detección de prompts
  moredelim=[s][\bfseries\color{ciscoblue}]{Switch>}{},
  moredelim=[s][\bfseries\color{ciscoblue}]{Switch\#}{},
  moredelim=[s][\bfseries\color{ciscoblue}]{SW1>}{},
  moredelim=[s][\bfseries\color{ciscoblue}]{SW1\#}{},
  moredelim=[s][\bfseries\color{ciscoblue}]{Router>}{},
  moredelim=[s][\bfseries\color{ciscoblue}]{Router\#}{}
}

\lstdefinestyle{custom-bash-style}{
  language=,
  basicstyle=\ttfamily\footnotesize\color{linuxwhite},
  backgroundcolor=\color{linuxbg},
  frame=none,
  numbers=none,
  showstringspaces=false,
  breaklines=true,
  breakatwhitespace=true,
  tabsize=2,
  captionpos=none,
  aboveskip=0pt,
  belowskip=0pt,
  xleftmargin=0pt,
  xrightmargin=0pt,
  framexleftmargin=0pt,
  framexrightmargin=0pt,
  % Comandos principales de Linux/Bash
  morekeywords={
    sudo, apt, update, install, which, ping, echo, cat, grep, head, tail,
    chmod, cp, mv, rm, ls, cd, mkdir, netstat, ss, ps, kill, top,
    systemctl, service, iptables, tcpdump, nmap, curl, wget,
    tshark, wireshark, macof, nc, if, then, else, fi, for, do, done,
    while, until, case, function, export, source, exit, return
  },
  keywordstyle=\bfseries\color{linuxcyan},
  % Opciones y flags
  morekeywords=[2]{
    -c, -i, -f, -u, -l, -y, -p, -n, -r, -4, -6, -v, -a, -h,
    --help, --version, -la, -lh, -rf, --color, --recursive
  },
  keywordstyle=[2]\bfseries\color{linuxyellow},
  % Strings
  stringstyle=\color{linuxpromptgreen},
  morestring=[b]",
  morestring=[b]',
  % Comentarios - usar ## para comentarios de script
  morecomment=[l]{\#\#},
  commentstyle=\color{linuxgray}\itshape,
  % Prompts de usuario con colores de terminal real
  moredelim=[s][\bfseries\color{linuxpromptgreen}]{ccna@}{:},
  moredelim=[s][\bfseries\color{linuxpromptblue}]{:~}{$},
  moredelim=[s][\bfseries\color{linuxpromptgreen}]{root@}{:},
  moredelim=[s][\bfseries\color{linuxpromptgreen}]{user@}{:},
  moredelim=[s][\bfseries\color{linuxpromptgreen}]{ubuntu@}{:}
}

% Caja de información
\newtcolorbox{info-box}{
  colback=infocolor!10,
  colframe=infocolor,
  arc=2pt,
  outer arc=2pt,
  leftrule=4pt,
  rightrule=0pt,
  toprule=0pt,
  bottomrule=0pt,
  left=8pt,
  right=8pt,
  top=8pt,
  bottom=8pt,
  before upper={\parindent15pt\noindent\textcolor{infocolor}{\Large\faInfoCircle}\enspace}
}

% Caja de advertencia
\newtcolorbox{warning-box}{
  colback=warningcolor!10,
  colframe=warningcolor,
  arc=2pt,
  outer arc=2pt,
  leftrule=4pt,
  rightrule=0pt,
  toprule=0pt,
  bottomrule=0pt,
  left=8pt,
  right=8pt,
  top=8pt,
  bottom=8pt,
  before upper={\parindent15pt\noindent\textcolor{warningcolor}{\Large\faExclamationTriangle}\enspace}
}

% Caja de error
\newtcolorbox{error-box}{
  colback=errorcolor!10,
  colframe=errorcolor,
  arc=2pt,
  outer arc=2pt,
  leftrule=4pt,
  rightrule=0pt,
  toprule=0pt,
  bottomrule=0pt,
  left=8pt,
  right=8pt,
  top=8pt,
  bottom=8pt,
  before upper={\parindent15pt\noindent\textcolor{errorcolor}{\Large\faTimes}\enspace}
}

% Caja de éxito
\newtcolorbox{success-box}{
  colback=successcolor!10,
  colframe=successcolor,
  arc=2pt,
  outer arc=2pt,
  leftrule=4pt,
  rightrule=0pt,
  toprule=0pt,
  bottomrule=0pt,
  left=8pt,
  right=8pt,
  top=8pt,
  bottom=8pt,
  before upper={\parindent15pt\noindent\textcolor{successcolor}{\Large\faCheck}\enspace}
}

% Caja para comandos Cisco IOS
\newtcolorbox{cisco-ios-box}{
  colback=ciscobg,
  colframe=ciscoframe,
  arc=4pt,
  outer arc=4pt,
  left=12pt,
  right=12pt,
  top=12pt,
  bottom=12pt,
  fonttitle=\bfseries\color{white},
  title={\faNetworkWired\enspace Cisco IOS Terminal},
  coltitle=white,
  colbacktitle=ciscoframe,
  boxed title style={arc=3pt},
  breakable,
  listing only,
  listing options={
    language=,
    basicstyle=\ttfamily\footnotesize\color{ciscotext},
    backgroundcolor=\color{ciscobg},
    frame=none,
    numbers=none,
    showstringspaces=false,
    breaklines=true,
    breakatwhitespace=true,
    tabsize=2,
    aboveskip=0pt,
    belowskip=0pt,
    xleftmargin=0pt,
    xrightmargin=0pt,
    % Palabras clave principales
    morekeywords={
      enable, configure, terminal, interface, ip, router, switch, vlan, 
      show, clear, debug, no, shutdown, exit, end, copy, reload, write,
      access, trunk, switchport, mode, spanning-tree, portfast,
      line, con, vty, login, password, secret, hostname, banner, motd,
      running-config, startup-config, memory, flash, tftp,
      port-security, maximum, violation, aging, sticky, mac-address,
      count, dynamic, static, learned, secure, restrict, protect
    },
    keywordstyle=\bfseries\color{ciscoblue},
    % Palabras clave secundarias
    morekeywords={[2]
      FastEthernet, GigabitEthernet, Serial, Loopback, VLAN,
      route, protocol, range, default-gateway, version, brief, detail,
      trunk, native, allowed, auto, full, half, dot1q, active, passive,
      UP, DOWN, administratively, connected, static, dynamic
    },
    keywordstyle={[2]\bfseries\color{ciscopurple}},
    % Strings
    stringstyle=\color{ciscored},
    morestring=[b]",
    morestring=[b]',
    % Comentarios
    morecomment=[l]{!},
    commentstyle=\color{ciscogreen}\itshape,
    % Prompts
    moredelim=[s][\bfseries\color{ciscoblue}]{Switch>}{},
    moredelim=[s][\bfseries\color{ciscoblue}]{Switch\#}{},
    moredelim=[s][\bfseries\color{ciscoblue}]{SW1>}{},
    moredelim=[s][\bfseries\color{ciscoblue}]{SW1\#}{},
    moredelim=[s][\bfseries\color{ciscoblue}]{Router>}{},
    moredelim=[s][\bfseries\color{ciscoblue}]{Router\#}{}
  }
}

% Caja para comandos Linux/Bash
% Definir estilo de listings para bash separadamente
\lstdefinestyle{linuxterminal}{
  language=,
  basicstyle=\ttfamily\footnotesize\color{linuxwhite},
  backgroundcolor=\color{linuxbg},
  frame=none,
  numbers=none,
  showstringspaces=false,
  breaklines=true,
  breakatwhitespace=true,
  tabsize=2,
  aboveskip=0pt,
  belowskip=0pt,
  xleftmargin=0pt,
  xrightmargin=0pt,
  % Comandos principales
  morekeywords={
    sudo, apt, update, install, which, ping, echo, cat, grep, head, tail,
    chmod, cp, mv, rm, ls, cd, mkdir, netstat, ss, ps, kill, top,
    systemctl, service, iptables, tcpdump, nmap, curl, wget,
    tshark, wireshark, macof, nc, if, then, else, fi, for, do, done,
    while, until, case, function, export, source, exit, return
  },
  keywordstyle=\bfseries\color{linuxcyan},
  % Opciones y flags
  morekeywords=[2]{
    -c, -i, -f, -u, -l, -y, -p, -n, -r, -4, -6, -v, -a, -h,
    --help, --version, -la, -lh, -rf, --color, --recursive
  },
  keywordstyle=[2]\bfseries\color{linuxyellow},
  % Strings
  stringstyle=\color{linuxpromptgreen},
  morestring=[b]",
  morestring=[b]',
  % Comentarios - usar ## para comentarios explícitos en scripts
  morecomment=[l]{\#\#},
  commentstyle=\color{linuxgray}\itshape,
  % Delimitadores para prompts de usuario
  moredelim=[is][\bfseries\color{linuxpromptgreen}]{[PROMPT]}{[/PROMPT]},
  moredelim=[is][\color{linuxpromptblue}]{[PATH]}{[/PATH]}
}

\newtcolorbox{bash-box}{
  enhanced,
  colback=linuxbg,
  colframe=linuxtitlebar,
  arc=6pt,
  outer arc=6pt,
  boxrule=1pt,
  left=10pt,
  right=10pt,
  top=8pt,
  bottom=8pt,
  fonttitle=\bfseries\footnotesize\color{white},
  title={\faTerminal\enspace Linux Terminal},
  coltitle=white,
  colbacktitle=linuxtitlebar,
  attach boxed title to top left={xshift=6pt, yshift=-\tcboxedtitleheight/2},
  boxed title style={
    arc=4pt,
    outer arc=4pt,
    boxrule=0pt,
    left=4pt,
    right=4pt
  },
  breakable,
  before upper={\color{linuxwhite}\ttfamily\footnotesize},
  after upper={},
  fontupper=\ttfamily\footnotesize,
  colupper=linuxwhite
}

% Hook para forzar estilos después de que Eisvogel cargue los suyos
\AtBeginDocument{%
  % Redefinir colores de listing para que bash-box funcione correctamente
  % Esto asegura que nuestros estilos personalizados tengan prioridad
}

% Forzar aplicación de estilos en entornos específicos
\BeforeBeginEnvironment{bash-box}{%
  \lstset{style=linuxterminal}%
}

\newcommand{\forceloadstyles}{%
  \lstset{style=custom-cisco-style}%
  \lstset{style=linuxterminal}%
}